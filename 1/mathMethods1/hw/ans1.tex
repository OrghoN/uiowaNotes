% \documentclass[oneside]{article}
\documentclass[12pt, oneside]{article}

\title{29:4761 - Homework Assignment \#1}
\author{Orgho Neogi}
\date{}

\usepackage{amssymb}
\usepackage{amsmath}
\makeatletter

\usepackage{graphicx}

\usepackage{float}

\usepackage[colorlinks=true]{hyperref}

\usepackage{enumitem}

\usepackage{epigraph}

\usepackage{csquotes}

\usepackage[english]{babel}

\usepackage{amsthm,thmtools}
\usepackage[nottoc]{tocbibind}

\usepackage{caption}

\usepackage{longtable}

\usepackage{physics}

\usepackage{listings}
\lstset{
basicstyle=\small\ttfamily,
columns=flexible,
breaklines=true
}

\newenvironment{answer}
  {\vspace*{0.2cm} \rule{12cm}{0.02cm} \vspace*{0.2cm}}
  {\vspace*{0.2cm}}

\newcommand*{\rom}[1]{\expandafter\@slowromancap\romannumeral #1@}

\setlength{\parskip}{1em}
% \renewcommand{\baselinestretch}{2.0}
% \usepackage[margin=1.2in]{geometry}

\begin{document}

\maketitle

\begin{enumerate}
  \item Prove

  \begin{equation*}
    e^{z_1}e^{z_2} = e^{z_1 + z_2}
  \end{equation*}

  for any pair of complex numbers $z_1$ and $z_2$. Use the definition

  \begin{equation*}
    e^z = 1 + z + \frac{z^2}{2!} + \frac{z^3}{3!} + \dots = \sum_{n=0}^{\infty} \frac{z^n}{n!}
  \end{equation*}

    \begin{answer}
      \begin{align*}
        e^{z_1}e^{z2} &= \sum_{i=0}^{\infty} \frac{z_1^i}{i!} \sum_{j=0}^{\infty} \frac{z_2^j}{j!} &&& \text{Definition of exponential} \\
         &= \sum_{k=0}^{\infty}\sum_{l=0}^{k} \frac{z_1^l}{l!}\frac{z_2^{k-l}}{(k-l)!} &&& \text{Cauchy product}\\
         &= \sum_{k=0}^{\infty} \frac{1}{k!} \sum_{l=0}^{k} \frac{k!}{l!(k-l)!}z_1^lz_2^{k-l} \\
         &= \sum_{k=0}^{\infty} \frac{(z_1 + z_2)^k}{k!} &&& \text{Binomial theorem} \\
         &= e^{z_1 + z_2} &&& \text{Definition of exponential} \\
      \end{align*}
    \end{answer}

  \item Find the real and imaginary parts of $e^z$

    \begin{answer}

      $z \in \mathbb{C}$, so $z$ can be written as $z = x + iy$, where $x, y \in \mathbb{R}$.

      \begin{align*}
        &e^z \\
        = &e^{x + iy} \\
        = &e^{x}e^{iy} \\
        = &e^{x}(\cos(y) + i\sin(y)) \\
        = &e^{x}\cos(y) + ie^{x}\sin(y)
      \end{align*}

      So, the real part is $e^{x}\cos(y)$ and the imaginary part is $e^{x}\sin(y)$
    \end{answer}

  \item Let $f(z)$ be a complex function of a complex variable. Show that $f(z)f(z)^*$ is always real and non-negative.

    \begin{answer}

      Since $f(z)$ is a complex function, $f(z)$ can be written as
        \begin{align*}
          f(z) &= x + iy \\
          f(z)^* &= x - iy\\
          \therefore f(z)f(z)^* & = (x + iy)(x - iy)\\
          &= x^2 -xiy +iyx -i^2y^2 \\
          &= x^2 +y^2
        \end{align*}

    $f(z)f(z)^*$ is real because there is no $i$ when the answer is simplified and it is always non negative since it is a sum of squares.

    \end{answer}

  \item Use the quadratic formula to factorize the following polynomial

  \begin{equation*}
    P(z) = z^2 + 3z + 12
  \end{equation*}


  Use complex arithmetic to verify that the polynomial is recovered by multiplying the factors.

    \begin{answer}

      Quadratic Formula:

      For equation $ax^2 + bx + c$,

      \begin{equation*}
        x = \frac{-b \pm \sqrt{b^2 - 4ac}}{2a}
      \end{equation*}

      So,

      \begin{align*}
        z &= \frac{-3 \pm \sqrt{3^2 - 4\cdot1\cdot12}}{2\cdot1} \\
          &= \frac{-3 \pm \sqrt{9 - 48}}{2}\\
          &= \frac{-3 \pm \sqrt{-39}}{2}\\
          &= \frac{-3 \pm i\sqrt{39}}{2}\\
          &= \frac{-3 - i\sqrt{39}}{2} \& \frac{-3 + i\sqrt{39}}{2}
      \end{align*}

      The factors are $(z + \frac{3 + i\sqrt{39}}{2})$ and $(z + \frac{3 - i\sqrt{39}}{2})$.If the factors are multiplied together we should get back to the original polynomial

      \begin{align*}
        P(z) &= (z + \frac{3 + i\sqrt{39}}{2})(z + \frac{3 - i\sqrt{39}}{2})\\
             &= z^2 + \frac{3-i\sqrt{39}}{2}z + \frac{3 + i\sqrt{39}}{2}z + (\frac{3 + i\sqrt{39}}{2})(\frac{3 - i\sqrt{39}}{2})\\
             &= z^2 + \frac{3z - i\sqrt{39}z + 3z + i\sqrt{39}z}{2} + \frac{9 + 3i\sqrt{39} - 3i\sqrt{39} + 39}{4}\\
             &= z^2 + \frac{6z}{2} + \frac{48}{4}\\
             &= z^2 + 3z + 12
      \end{align*}

    \end{answer}

  \item Calculate the real and imaginary parts of

  \begin{equation*}
    \frac{10+i5}{7-i6}
  \end{equation*}

    \begin{answer}

      \begin{align*}
        &\frac{10+i5}{7-i6}\\
        = &\frac{10+i5}{7-i6}\cdot\frac{7+i6}{7+i6}\\
        = &\frac{70 + i35 + i60 + i^2\cdot30}{7^2 -i6\cdot7 + 7\cdot i6 -i^2\cdot6^2}\\
        = &\frac{40+95i}{49+36}\\
        =&\frac{40+95i}{85}\\
        =&\frac{8}{17} + \frac{19}{17}i
      \end{align*}

      So, the real part is $\frac{8}{17}$ and the imaginary part is $\frac{19}{17}$
    \end{answer}

  \item Find the modulus and argument of $cos(ix)$

    \begin{answer}

      \begin{align*}
        \cos(ix) &= \cosh(x)\\
                 &= \frac{e^x+e^{-x}}{2}
      \end{align*}

      So, $\cos(ix)$ has only a real part, which means that the modulus is $\frac{e^x+e^{-x}}{2}$ and the argument is 0 since the angle $\theta$ is 0.
    \end{answer}


\end{enumerate}

\end{document}
