% \documentclass[oneside]{article}
\documentclass[12pt, oneside]{article}

\title{29:4761 - Homework Assignment \#2}
\author{Orgho Neogi}
\date{}

\usepackage{amssymb}
\usepackage{amsmath}
\makeatletter

\usepackage{graphicx}

\usepackage{float}

\usepackage[colorlinks=true]{hyperref}

\usepackage{enumitem}

\usepackage{epigraph}

\usepackage{csquotes}

\usepackage[english]{babel}

\usepackage{amsthm,thmtools}
\usepackage[nottoc]{tocbibind}

\usepackage{caption}

\usepackage{longtable}

\usepackage{physics}

\usepackage{listings}
\lstset{
basicstyle=\small\ttfamily,
columns=flexible,
breaklines=true
}

\newenvironment{answer}
  {\vspace*{0.2cm} \rule{12cm}{0.02cm} \vspace*{0.2cm}}
  {\vspace*{0.2cm}}

\newcommand*{\rom}[1]{\expandafter\@slowromancap\romannumeral #1@}

\setlength{\parskip}{1em}
% \renewcommand{\baselinestretch}{2.0}
% \usepackage[margin=1.2in]{geometry}

\allowdisplaybreaks

\begin{document}

\maketitle

\begin{enumerate}
  \item Show that if $f_1(z) = u_1(x,y) + iv_1(x,y)$ and $f_2(z) = u_2(x,y) + iv_2(x,y)$ have continuous partial derivatives and satisfy the Cauchy-Riemann equations that $g(z) = f_1(f_2(z))$ also has continuous partial derivatives and satisfies the Cauchy-Riemann equations.

  \begin{answer}

    If $g(z)$ satisfies the Cauchy-Riemann equations, then it has continuous partial derivatives.

    \begin{align*}
      g(z) &= f_1(f_2(z))\\
           &= f_1(u_2(x,y) + iv_2(x,y))\\
           &= u_1(u_2(x,y),v_2(x,y)) + iv_1(u_2(x,y),v_2(x,y))
    \end{align*}

    Let $u = u_1(u_2(x,y),v_2(x,y))$ and $v = v_1(u_2(x,y),v_2(x,y))$

    \begin{align*}
      \frac{du}{dx} &= \frac{du_1}{du_2}\frac{du_2}{dx} +\frac{du_1}{dv_2}\frac{dv_2}{dx}\\
      \frac{dv}{dy} &= \frac{dv_1}{du_2}\frac{du_2}{dy} +\frac{dv_1}{dv_2}\frac{dv_2}{dy}\\
                    &= -\frac{du_1}{dv_2}\left(-\frac{dv_2}{dx}\right) +\frac{du_1}{du_2}\frac{du_2}{dx}\\
                    &= \frac{du_1}{dv_2}\frac{dv_2}{dx}+\frac{du_1}{du_2}\frac{du_2}{dx}\\
      \frac{dv}{dy} &= \frac{du}{dx}\\
      \frac{du}{dy} &=\frac{du_1}{du_2}\frac{du_2}{dy} +\frac{du_1}{dv_2}\frac{dv_2}{dy}\\
      \frac{dv}{dx} &= \frac{dv_1}{du_2}\frac{du_2}{dx} +\frac{dv_1}{dv_2}\frac{dv_2}{dx}\\
                    &= -\frac{du_1}{dv_2}\frac{dv_2}{dy} +\frac{du_1}{du_2}\left(-\frac{du_2}{dy}\right)\\
                    &= -\left(\frac{du_1}{dv_2}\frac{dv_2}{dy} +\frac{du_1}{du_2}\frac{du_2}{dy}\right)\\
      \frac{du}{dy} &= -\frac{dv}{dx}
    \end{align*}
  \end{answer}

  \item Find the real and imaginary parts of $\sin(z) = u(x,y) + iv(x,y)$ and show they are solutions of Laplace’s equation and the gradients, of each function are orthogonal, $\gradient u \cdot \gradient v = 0$.

  \begin{answer}
    \begin{align*}
      z &= x + iy\\
      \sin(z) &= \frac{e^{iz}-e^{-iz}}{2i}\\
      &= \frac{e^{i(x+iy)}-e^{-i(x+iy)}}{2i}\\
      &= \frac{e^{ix-y}-e^{-ix+y}}{2i}\\
      &= \frac{e^{ix}e^{-y}-e^{-ix}e^{y}}{2i}\\
      &= \frac{(\cos(x)+i\sin(x))e^{-y}-(\cos(x)-i\sin(x))e^{y}}{2i}\\
      &= \frac{e^{-y}\cos(x)+ie^{-y}\sin(x)-e^{y}\cos(x)-ie^{y}\sin(x)}{2i}\cdot\frac{i}{i}\\
      &= \frac{-ie^{-y}\cos(x)+e^{-y}\sin(x)+ie^{y}\cos(x)+e^{y}\sin(x)}{2}\\
      &= \sin(x)(\frac{e^y + e^-y}{2}) + i\cos(x)(\frac{e^y - e^-y}{2})\\
      &= \sin(x)\cosh(y) + i\cos(x)\sinh(y)\\
    \end{align*}

    So, $u = \sin(x)\cosh(y)$ and $v = \cos(x)\sinh(y)$\label{breakdown}

    Laplace's equation states that $\divergence \gradient f = 0$ In this case

    \begin{align*}
      \divergence \gradient \sin(z) &= \frac{\partial^2}{\partial x^2}\sin(z) + \frac{\partial^2}{\partial y^2}\sin(z)
    \end{align*}

    \begin{align*}
      \frac{\partial^2}{\partial x^2}\sin(z) &= \frac{\partial}{\partial x}\frac{\partial}{\partial x}\sin(z)\\
      &= \frac{\partial}{\partial x}\frac{\partial}{\partial x}(\sin(x)\cosh(y) + i\cos(x)\sinh(y))\\
      &= \frac{\partial}{\partial x}(\cos(x)\cosh(y) - i\sin(x)\sinh(y))\\
      &= -\sin(x)\cosh(y) - i\cos(x)\sinh(y)\\
    \end{align*}
    \begin{align*}
      \frac{\partial^2}{\partial y^2}\sin(z) &= \frac{\partial}{\partial y}\frac{\partial}{\partial y}\sin(z)\\
      &= \frac{\partial}{\partial y}\frac{\partial}{\partial y}(\sin(x)\cosh(y) + i\cos(x)\sinh(y))\\
      &= \frac{\partial}{\partial y}(\sin(x)\sinh(y) + i\cos(x)\cosh(y))\\
      &= \sin(x)\cosh(y) + i\cos(x)\sinh(y)\\
    \end{align*}

    \begin{align*}
      \divergence \gradient \sin(z) &= -\sin(x)\cosh(y) - i\cos(x)\sinh(y) + \sin(x)\cosh(y) + i\cos(x)\sinh(y)\\
       &= 0
    \end{align*}

    % $\sin(z)$ fulfills laplaces equation

    \begin{align*}
      \gradient u &= \begin{bmatrix}
                     & \frac{\partial}{\partial x}u & \frac{\partial}{\partial y}u &
                     \end{bmatrix}\\
                  &= \begin{bmatrix}
                     & \cos(x)\cosh(y) & \sin(x)\sinh(y) &
                     \end{bmatrix}\\
       \gradient v &= \begin{bmatrix}
                      & \frac{\partial}{\partial x}v & \frac{\partial}{\partial y}v &
                      \end{bmatrix}\\
                   &= \begin{bmatrix}
                      & -\sin(x)\sinh(y) & \cos(x)\cosh(y) &
                      \end{bmatrix}\\
        \gradient u \cdot \gradient v &= \begin{bmatrix}
           & \cos(x)\cosh(y) & \sin(x)\sinh(y) &
           \end{bmatrix} \cdot\begin{bmatrix}
              & -\sin(x)\sinh(y) & \\ & \cos(x)\cosh(y) &
              \end{bmatrix}\\
              &= \cos(x)\cosh(y)(-\sin(x)\sinh(y)) + \sin(x)\sinh(y)(\cos(x)\cosh(y))\\
              &= -\cos(x)\cosh(y)\sin(x)\sinh(y) + \sin(x)\sinh(y)\cos(x)\cosh(y)\\
              &=0
    \end{align*}

  \end{answer}

  \item For $z = re^{i\theta} = x + iy$, let $f(z) = u(r,\theta) + iv(r,\theta)$. Derive the form of the Cauchy-Riemann equations in $r$,$\theta$ variables.

  \begin{answer}

    Assuming that $f(z)$ fulfills the Cauchy-Riemann equations in the first place, implying that

  \begin{align*}
      \frac{\partial u}{\partial x} = \frac{\partial v}{\partial y} \quad \quad
      \frac{\partial u}{\partial y} = -\frac{\partial v}{\partial x}
  \end{align*}

  Converting between $x,y$ basis and $r,\theta$ basis
  \begin{align*}
    z &= re^{i\theta}\\
    z &= r(\cos(\theta) + i\sin(\theta))\\
    \therefore x &= r\cos(\theta)\\
    \therefore y &= r\sin(\theta)\\
  \end{align*}

  Differentiating to find the $r,\theta$ form of the Cauchy-Riemann equations

  \begin{align*}
    \frac{\partial u}{\partial \theta} &= \frac{\partial u}{\partial x}r\sin(\theta) - \frac{\partial u}{\partial y}r\cos(\theta)\\
    \frac{\partial v}{\partial \theta} &= \frac{\partial v}{\partial x}r\sin(\theta) + \frac{\partial v}{\partial y}r\cos(\theta)\\
    \frac{\partial u}{\partial r} &= \frac{\partial u}{\partial x}\cos(\theta) +\frac{\partial u}{\partial y}\sin(\theta)\\
     &= \frac{1}{r}(\frac{\partial v}{\partial y}r\cos(\theta) +\frac{\partial v}{\partial x}r\sin(\theta))\\
     &= \frac{1}{r}\frac{\partial v}{\partial \theta}\\
     \frac{\partial v}{\partial r} &= \frac{\partial v}{\partial x}\cos(\theta) +\frac{\partial v}{\partial y}\sin(\theta)\\
      &= -\frac{1}{r}(\frac{\partial u}{\partial y}r\cos(\theta) -\frac{\partial u}{\partial x}\sin(\theta))\\
      &= -\frac{1}{r}\frac{\partial u}{\partial \theta}\\
  \end{align*}

  So,

  \begin{align*}
      \frac{\partial u}{\partial r} = \frac{1}{r}\frac{\partial v}{\partial \theta} \quad \quad
      \frac{\partial u}{\partial \theta} = -r\frac{\partial v}{\partial r}
  \end{align*}

  \end{answer}

  \item Using the definition, show that $f(z) = (a - z)/(b - z)$, has a complex derivative.

  \begin{answer}

    A function $f$ is differentiable at a point $z_0$ if the limit

    \begin{align*}
      \lim_{h\to0} \frac{f(z_0 + h) - f(z_0)}{h}
    \end{align*}

    exists finitely

    the function $f(z) = (a - z)/(b - z)$ is differentiable if

    \begin{align*}
      \lim_{h\to0} \frac{\frac{a - z_0 - h}{b - z_0 - h} - \frac{a - z_0}{b - z_0}}{h}
    \end{align*}

    converges

    \begin{align*}
      &\lim_{h\to0} \frac{\frac{(a-z_0-h)(b-z_0) - (a-z_0)(b-z_0-h)}{(b-z_0)(b - z_0 - h)}}{h}\\
      =&\lim_{h\to0} \frac{\frac{ah - bh}{b^2 - bz_0 -bh - bz_0 +z_0^2 + hz_0}}{h}\\
      =&\lim_{h\to0} \frac{a - b}{b^2 - bz_0 -bh - bz_0 +z_0^2 + hz_0}\\
      =&\frac{a - b}{b^2 - 2bz_0 +z_0^2}\\
    \end{align*}

    Since there is a finite limit, the function is complex differentiable everywhere
  \end{answer}

  \item Let $u(x,y) = ax^3 + bx^2y + cxy^2 + dy^3$. Find values of $a$, $b$, $c$, $d$ for which this function satisfies Laplace’s equation. For this $u(x,y)$ find a corresponding $v(x,y)$ such that $u(x,y)$ and $v(x,y)$ satisfy the Cauchy-Riemann equations.

  \begin{answer}

    Discounting the trivial case where $a,b,c,d$ are all equal to $0$, to satisfy laplaces equation, $\divergence \gradient u = 0$ So,

    \begin{align*}
      \frac{\partial^2}{\partial x^2}u + \frac{\partial^2}{\partial y^2}u &= 0\\
      \frac{\partial^2}{\partial x^2}u &= - \frac{\partial^2}{\partial y^2}u\\
      \frac{\partial^2}{\partial x^2}u &=\frac{\partial}{\partial x}\frac{\partial}{\partial x}u\\
      &=\frac{\partial}{\partial x}\frac{\partial}{\partial x}(ax^3 + bx^2y + cxy^2 + dy^3)\\
      &=\frac{\partial}{\partial x}(3ax^2+2bxy+cy^2)\\
      &=6ax+2by\\
      \frac{\partial^2}{\partial y^2}u &=\frac{\partial}{\partial y}\frac{\partial}{\partial y}u\\
      &=\frac{\partial}{\partial y}\frac{\partial}{\partial y}u\\
      &=\frac{\partial}{\partial y}\frac{\partial}{\partial y}(ax^3 + bx^2y + cxy^2 + dy^3)\\
      &=\frac{\partial}{\partial y}(bx^2+2cxy+3dy^2)\\
      &=2cx+6dy
    \end{align*}

    This implies that

    \begin{align*}
      6a &= -2c\\
      c &= -3a\\
      2b &= -6d\\
      b &= -3d
    \end{align*}

    So one set of values that would work would be $a=1$,$b=-3$,$c=-3$,$d=1$
    This particular version of $u(x,y) = x^3 -3x^2y -3xy^2 + y^3$

    For this version of $u(x,y)$
    \begin{align*}
      \frac{\partial u}{\partial x} &= 3x^2-6xy-3y^2 = \frac{\partial v}{\partial y}\\
      \frac{\partial u}{\partial y} &= 3y^2-3x^2-6xy = -\frac{\partial v}{\partial x}\\
    \end{align*}

    To find v we can integrate
    \begin{align*}
      v &= \int\frac{\partial v}{\partial x}\partial x\\
        &= -\int\frac{\partial u}{\partial y}\partial x\\
        &= -\int (3y^2-3x^2-6xy) \partial x\\
        &= x^3 +3x^2y-3xy^2 + c_1\\
      v &= \int\frac{\partial v}{\partial y}\partial y\\
        &= \int\frac{\partial u}{\partial x}\partial y\\
        &= \int(3x^2-6xy-3y^2)\partial y\\
        &= 3x^2y-3xy^2-y^3 + c_2\\
      \therefore v &= x^3 + 3x^2y - 3xy^2 - y^3
    \end{align*}
  \end{answer}

  \item Calculate the integral of $f(z) = \sin(z)$ from $z=0$ to $z=1 + i$ first along the straight line paths from $z = 0$ to $z = 1$ then from $z = 1$ to $z = 1 + i$. Next calculate the integral of the same function along the straight line path from $z = 0$ to $z = 1 + i$. Show that both integrals give the same result.

  \begin{answer}

    As shown in answer \ref{breakdown}, $f(z) = \sin(z) = \frac{e^{ix-y}-e^{-ix+y}}{2i}$ where $x$ \& $y$ are real numbers. To first calculate the integral along he straight line paths from $z = 0$ to $z = 1$ then from $z = 1$ to $z = 1 + i$, we can break the path down as follows.
    For the first part, $y=0$, so $z=x$ and $dz = dx$, $0\leq x \leq 1$. For the second   part, $x=1$, so $z = 1 + iy$ and $dz = idy$,$0\leq y \leq 1$.

    The integral breaks down into:

    \begin{align*}
      &\int_0^1 \frac{e^{ix}-e^{-ix}}{2i} dx+ \int_0^1 \frac{e^{i-y}-e^{-i+y}}{2i}(idy)\\
      =&\frac{1}{2i}\int_0^1 e^{ix}-e^{-ix} dx+ \frac{i}{2i} \int_0^1 e^{i-y}-e^{-i+y}dy\\
      =&\frac{1}{2i}(-ie^{ix}-ie^{-ix})\Big|_0^{1}+ \frac{1}{2} (-e^{i-y}-e^{-i+y})\Big|_0^{1}\\
      =&\frac{1}{2}(2-e^{i-1}-e^{1-i})\\
      =&1-\frac{e^{i-1}+e^{1-i}}{2}\\
      =&1-\cos(1+i)
    \end{align*}

    When integrating along the straight line $y=x$, $z=x+ix=(1+i)x$, so $dz=(1+i)dx$, $0\leq x \leq 1$. This means the integral is

    \begin{align*}
      &\int_0^{1} \frac{e^{ix-x}-e^{-ix+x}}{2i} (1+i)dx\\
      =&\frac{1+i}{2}\int_0^{1} e^{(i-1)x}-e^{(1-i)x}dx\\
      =&\left.\frac{1+i}{2}\left[\left(-\frac{1}{2}-\frac{i}{2}\right) e^{(i-1)x}-\left(\frac{1}{2}+\frac{i}{2}\right)e^{(1-i)x}\right]\right|_0^{1}\\
      =&\left.-\frac{1}{2}\left(e^{(i-1)x}+e^{(1-i)x}\right)\right|_0^{1}\\
      =&1-\frac{e^{i-1}+e^{1-i}}{2}\\
      =&1-\cos(1+i)
    \end{align*}

  \end{answer}
\end{enumerate}

\end{document}
