% \documentclass[oneside]{article}
\documentclass[12pt, oneside]{article}

\title{PHYS:5741 \\Problem Set 2 \\Due Wednesday, September 11}
\author{Orgho Neogi}
\date{}

\usepackage{amssymb}
\usepackage{amsmath}
\makeatletter

\usepackage{graphicx}

\usepackage{float}

\usepackage[colorlinks=true]{hyperref}

\usepackage{enumitem}

\usepackage{epigraph}

\usepackage{csquotes}

\usepackage[english]{babel}

\usepackage{amsthm,thmtools}
\usepackage[nottoc]{tocbibind}

\usepackage{caption}

\usepackage{longtable}

\usepackage{physics}

\usepackage{listings}
\lstset{
basicstyle=\small\ttfamily,
columns=flexible,
breaklines=true
}

\newenvironment{answer}
  {\vspace*{0.2cm} \rule{12cm}{0.02cm} \vspace*{0.2cm}}
  {\vspace*{0.2cm}}

\newcommand*{\rom}[1]{\expandafter\@slowromancap\romannumeral #1@}

\setlength{\parskip}{1em}
% \renewcommand{\baselinestretch}{2.0}
% \usepackage[margin=1.2in]{geometry}

\begin{document}

\maketitle

  \begin{enumerate}
    \item Suppose a $2 \times 2$ matrix $X$ (not necessarily Hermitian or unitary) is written as

    \begin{align*}
      X = a_0 + \boldsymbol{\sigma \cdot a},
    \end{align*}

    Where $a_0$ and $a_{1,2,3}$ are numbers

    \begin{enumerate}
      \item How are $a_0$ and $a_k$($k=1,2,3$) related to Tr($X$) and Tr($\sigma_k X$)?

      \begin{answer}
        \begin{align*}
          \Tr(X) &= a_0\Tr(1) + \sum_l\Tr(\sigma_l)a_l\\
                 &= 2a_0\\
          \Tr(\sigma_kX)&= a_0\Tr(\sigma_k) + \sum_l\Tr(\sigma_k\sigma_l)a_l\\
                        &= \frac{1}{2}\sum_l\Tr(\sigma_k\sigma_l + \sigma_l\sigma_k)a_l\\
                        &= \sum_l\delta_{kl}\Tr(1)a_l\\
                        &= 2a_k
        \end{align*}
      \end{answer}

      \item Obtain $a_0$ and $a_k$ in terms of the matrix elements $X_{ij}$

      \begin{answer}
        \begin{align*}
          a_0 &= \frac{X_{11}+X_{22}}{2}\\
          a_1 &= \frac{X_{12}+X_{21}}{2}\\
          a_2 &= \frac{i(X_{12}-X_{21})}{2}\\
          a_3 &= \frac{X_{11}-X_22}{2}\\
        \end{align*}
      \end{answer}
    \end{enumerate}

    \item The Hamiltonian operator for a two-state system is given by

    \begin{align*}
      H = a(\ket{1}\bra{1} - \ket{2}\bra{2} + \ket{1}\bra{2} + \ket{2}\bra{1})
    \end{align*}

    where $a$ is a number with the dimensions of energy. Find the energy eigenvalues and corresponding energy eigenkets (as linear combinations of $\ket{1}$ and $\ket{2}$)

    \begin{answer}
      \begin{align*}
        H &= \begin{bmatrix}
                & a & a  & \\
                & a & -a & \\
              \end{bmatrix}
      \end{align*}

      Characteristic equation:

      \begin{align*}
        0 &= (a - \lambda)(-a - \lambda) - a^2\\
        0 &= \lambda^2 - 2a^2 \\
        \therefore \lambda &= a\sqrt{2} \quad \& \; \; -a\sqrt{2}
      \end{align*}

      For $\lambda = a\sqrt{2}:$

      \begin{align*}
        (1 - \sqrt{2})x_1 + x_2 = 0
      \end{align*}

      For $\lambda = -a\sqrt{2}:$

      \begin{align*}
        (1 + \sqrt{2})x_1 + x_2 = 0
      \end{align*}

      Where $x_1$ \& $x_2$ are the 2 elements of the eigenvector. So, eigenkets are

      \begin{align*}
        \sqrt{\frac{1}{4-2\sqrt{2}}}\begin{bmatrix}
                                   & 1 &\\
                                   & \sqrt{2} - 1
                                   \end{bmatrix}
                                   \quad \& \; \;
         \sqrt{\frac{1}{4+2\sqrt{2}}}\begin{bmatrix}
                                    & -1 &\\
                                    & \sqrt{2} + 1
                                    \end{bmatrix}
      \end{align*}

      respectively

    \end{answer}

    \item A certain observable in quantum mechanics has a $3 \times 3$ matrix representation as follows:

    \begin{align*}
      \frac{1}{\sqrt{2}}\begin{bmatrix}
                          & 0 & 1 & 0 & \\
                          & 1 & 0 & 1 & \\
                          & 0 & 1 & 0 & \\
                        \end{bmatrix}
    \end{align*}

    \begin{enumerate}
      \item Find the normalized eigenvectors of this observable and the corresponding eigenvalues. Is there any degeneracy?

      \begin{answer}
        Characteristic Equation:
        \begin{align*}
          0 &= -\lambda^3 -2(-\lambda)(\frac{1}{\sqrt{2}})^2\\
          0 &= -\lambda(1-\lambda)^2 \\
          \therefore \lambda &= 0 \quad \& \quad 1 \quad \& \; \; -1
        \end{align*}

        So, there is no degeneracy. The eigenvectors are:
        \begin{align*}
          \frac{1}{\sqrt{2}}\begin{bmatrix}
                              & -1 & \\
                              & 0  & \\
                              & 1  & \\
                            \end{bmatrix}
                            \; \; \& \quad
          \frac{1}{2}\begin{bmatrix}
                              & 1 & \\
                              & \sqrt{2}  & \\
                              & 1  & \\
                            \end{bmatrix}
                            \; \; \& \quad
          \frac{1}{2}\begin{bmatrix}
                              & 1 & \\
                              & -\sqrt{2}  & \\
                              & 1  & \\
                            \end{bmatrix}
        \end{align*}

        respectively.
      \end{answer}

      \item Give a physical example where all this is relevant.

      \begin{answer}
        The eigenvectors are for a spin 1 system.
      \end{answer}
    \end{enumerate}

    \item Consider a three-dimensional ket space. If a certain set of orthonormal kets—say, $\ket{1}$, $\ket{2}$, $\ket{3}$ --- are used as base kets, the operators A and B are represented by

    \begin{align*}
      A = \begin{bmatrix}
            & a & 0  & 0  & \\
            & 0 & -a & 0  & \\
            & 0 & 0  & -a & \\
          \end{bmatrix}
      \quad B = \begin{bmatrix}
                  & b & 0  & 0   & \\
                  & 0 & 0  & -ib & \\
                  & 0 & ib & 0   & \\
                \end{bmatrix}
    \end{align*}

    with $a$ and $b$ both real

    \begin{enumerate}
      \item Obviously $A$ exhibits a degenerate spectrum. Does $B$ also exhibit a degenerate spectrum?

      \begin{answer}
        Characteristic equation for B:

        \begin{align*}
          0 &= (b-\lambda)(\lambda)^2-(b-\lambda)(ib)(-ib)\\
          0 &= (b-\lambda)(\lambda^2 - b^2) \\
          \therefore \lambda &= b \quad \& \quad b \quad \& \; \; -b
        \end{align*}

        Since $b$ repeats as an eigenvalue, they are degenerate
      \end{answer}

      \item Show that $A$ and $B$ commute.

      \begin{answer}
        \begin{align*}
          AB &= \begin{bmatrix}
                & a & 0  & 0  & \\
                & 0 & -a & 0  & \\
                & 0 & 0  & -a & \\
              \end{bmatrix}
          \cdot \begin{bmatrix}
                      & b & 0  & 0   & \\
                      & 0 & 0  & -ib & \\
                      & 0 & ib & 0   & \\
                    \end{bmatrix}\\
            &= \begin{bmatrix}
                        & ab & 0  & 0   & \\
                        & 0 & 0  & -iab & \\
                        & 0 & iab & 0   & \\
                      \end{bmatrix}\\
        BA &= \begin{bmatrix}
                    & b & 0  & 0   & \\
                    & 0 & 0  & -ib & \\
                    & 0 & ib & 0   & \\
                  \end{bmatrix}
              \cdot \begin{bmatrix}
                    & a & 0  & 0  & \\
                    & 0 & -a & 0  & \\
                    & 0 & 0  & -a & \\
                  \end{bmatrix}\\
                  &= \begin{bmatrix}
                              & ab & 0  & 0   & \\
                              & 0 & 0  & -iab & \\
                              & 0 & iab & 0   & \\
                            \end{bmatrix}\\
        \end{align*}

        Since $AB = BA$, $AB - BA = 0$, therefore they commute
      \end{answer}

      \item Find a new set of orthonormal kets that are simultaneous eigenkets of both $A$ and $B$. Specify the eigenvalues of $A$ and $B$ for each of the three eigenkets. Does your specification of eigenvalues completely characterize each eigenket?

      \begin{answer}
        Since A and B commute, they must share eigenvectors. Because A and B are both degenerate, the eigenvalues cannot characterize the eigenkets. One set of such simulatenout eigenkets are:

        For Eigenvalues of $a$ \& $b$: $\ket{1}$

        For Eigenvalues of $-a$ \& $-b$: $\frac{1}{\sqrt{2}}(\ket{2}+i\ket{3})$
        
        For Eigenvalues of $-a$ \& $-b$: $\frac{1}{\sqrt{2}}(\ket{2}-i\ket{3})$
      \end{answer}
    \end{enumerate}
  \end{enumerate}

\end{document}
