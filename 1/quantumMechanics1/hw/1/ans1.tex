% \documentclass[oneside]{article}
\documentclass[12pt, oneside]{article}

\title{PHYS:5741 \\Problem Set 1 \\Due Wednesday, September 4}
\author{Orgho Neogi}
\date{}

\usepackage{amssymb}
\usepackage{amsmath}
\makeatletter

\usepackage{graphicx}

\usepackage{float}

\usepackage[colorlinks=true]{hyperref}

\usepackage{enumitem}

\usepackage{epigraph}

\usepackage{csquotes}

\usepackage[english]{babel}

\usepackage{amsthm,thmtools}
\usepackage[nottoc]{tocbibind}

\usepackage{caption}

\usepackage{longtable}

\usepackage{physics}

\usepackage{listings}
\lstset{
basicstyle=\small\ttfamily,
columns=flexible,
breaklines=true
}

\newenvironment{answer}
  {\vspace*{0.2cm} \rule{12cm}{0.02cm} \vspace*{0.2cm}}
  {\vspace*{0.2cm}}

\newcommand*{\rom}[1]{\expandafter\@slowromancap\romannumeral #1@}

\setlength{\parskip}{1em}
% \renewcommand{\baselinestretch}{2.0}
% \usepackage[margin=1.2in]{geometry}

\begin{document}

\maketitle

For photons propagating in the z-direction the linearly polarized states polarized in the x and y directions are

\begin{align*}
  \ket{x} = \begin{bmatrix}1 \\  0\end{bmatrix} \; \ket{y} = \begin{bmatrix}0 \\  1\end{bmatrix}
\end{align*}

\noindent and the circularly polarized photon states are

\begin{align*}
  \ket{R} = \frac{1}{\sqrt{2}}\begin{bmatrix}1 \\  i\end{bmatrix} \; \ket{L} = \frac{1}{\sqrt{2}}\begin{bmatrix}1 \\  -i\end{bmatrix}
\end{align*}

\begin{enumerate}
  \item
  \begin{enumerate}
    \item Write down a basis that is neither plane nor circularly polarized \label{basis}

    \begin{answer}
      \begin{align*}
        \ket{a} = \begin{bmatrix}1 \\  0\end{bmatrix} \; \ket{b} = \frac{1}{\sqrt{2}}\begin{bmatrix}1 \\  1\end{bmatrix}
      \end{align*}

      A basis is formed if the set of vectors are linearly independent. This can be checked by writing it as a matrix

      \begin{align*}
        M = \begin{bmatrix}&1 &\frac{1}{\sqrt{2}}\\ &0 &\frac{1}{\sqrt{2}}\end{bmatrix}
      \end{align*}
    \end{answer}

    and then taking it's determinant. If the determinant is non-zero, then it forms a basis.

    \begin{align*}
      \abs{M} &= 1 \cdot \frac{1}{\sqrt{2}} - \frac{1}{\sqrt{2}} \cdot 0 \\
              &= \frac{1}{\sqrt{2}}
    \end{align*}

    \item Calculate the transformation matrix from the $x$, $y$ basis to the $R$, $L$ basis. \label{transformation1}

    \begin{answer}
      To go from $x$, $y$ basis to the $R$, $L$ basis, we can use

      \begin{align*}
        T_1 = \begin{bmatrix}
                &\bra{R}\ket{x} &\bra{R}\ket{y} &\\
                &\bra{L}\ket{x} &\bra{L}\ket{y} &
              \end{bmatrix}
      \end{align*}

      where

      \begin{align*}
        \ket{R} &= \frac{1}{\sqrt{2}}\begin{bmatrix}1 \\  i\end{bmatrix} \\
                &= \frac{1}{\sqrt{2}}(\begin{bmatrix}1 \\  0\end{bmatrix} + i \begin{bmatrix}0 \\  1\end{bmatrix})\\
                &= \frac{1}{\sqrt{2}}(\ket{x} + i\ket{y})\\
        \ket{L} &= \frac{1}{\sqrt{2}}\begin{bmatrix}1 \\  -i\end{bmatrix} \\
                &= \frac{1}{\sqrt{2}}(\begin{bmatrix}1 \\  0\end{bmatrix} - i \begin{bmatrix}0 \\  1\end{bmatrix})\\
                &= \frac{1}{\sqrt{2}}(\ket{x} - i\ket{y})\\
      \end{align*}

      So,

      \begin{align*}
        T_1 = \frac{1}{\sqrt{2}}\begin{bmatrix}
                                  &1 &-i &\\
                                  &1 &i &
                                \end{bmatrix}
      \end{align*}
    \end{answer}

    \item Calculate the transformation matrix from the $R$, $L$ basis to the basis you cooked up in part \ref{basis}. \label{transformation2}

    \begin{answer}
      To go from $R$, $L$ basis to the $a$, $b$ basis, we can use

      \begin{align*}
        T_2 = \begin{bmatrix}
                &\bra{a}\ket{R} &\bra{a}\ket{L} &\\
                &\bra{b}\ket{R} &\bra{b}\ket{L} &
              \end{bmatrix}
      \end{align*}

      where

      \begin{align*}
        \ket{a} &= \begin{bmatrix}1 \\  0\end{bmatrix} \\
                &= \frac{1}{\sqrt{2}}\cdot\frac{1}{\sqrt{2}}\begin{bmatrix}1 \\  i\end{bmatrix} + \frac{1}{\sqrt{2}}\cdot\frac{1}{\sqrt{2}}\begin{bmatrix}1 \\  -i\end{bmatrix})\\
                &= \frac{1}{\sqrt{2}}\ket{R} + \frac{1}{\sqrt{2}}\ket{L}\\
        \ket{b} &= \frac{1}{\sqrt{2}}\begin{bmatrix}1 \\  1\end{bmatrix}\\
                &= \frac{1 - i}{2}\cdot\frac{1}{\sqrt{2}}\begin{bmatrix}1 \\  i\end{bmatrix} + \frac{1 + i}{2}\cdot\frac{1}{\sqrt{2}}\begin{bmatrix}1 \\  -i\end{bmatrix})\\
                &= \frac{1-i}{2}\ket{R} + \frac{1+i}{2}\ket{L}
      \end{align*}

      So,

      \begin{align*}
        T_2 = \begin{bmatrix}
                &\frac{1}{\sqrt{2}} &\frac{1}{\sqrt{2}} &\\
                &\frac{1+i}{2} &\frac{1-i}{2} &
              \end{bmatrix}
      \end{align*}

    \end{answer}

    \item Calculate the transformation matrix from the $x$, $y$ basis to the basis in part \ref{basis},and show that it is the product of the matrices calculated in parts \ref{transformation1} and \ref{transformation2}. In which order must you multiply these matrices?

    \begin{answer}
      To go from $x$, $y$ basis to the $a$, $b$ basis, we can use

      \begin{align*}
        T_3 = \begin{bmatrix}
                &\bra{a}\ket{x} &\bra{a}\ket{y} &\\
                &\bra{b}\ket{x} &\bra{b}\ket{y} &
              \end{bmatrix}
      \end{align*}

      where

      \begin{align*}
        \ket{a} &= \begin{bmatrix}1 \\  0\end{bmatrix} \\
                &= 1\begin{bmatrix}1 \\  0\end{bmatrix} + 0\begin{bmatrix}0 \\ 1\end{bmatrix})\\
                &= 1\ket{x} + 0\ket{y}\\
        \ket{b} &= \frac{1}{\sqrt{2}}\begin{bmatrix}1 \\  1\end{bmatrix}\\
                &= \frac{1}{\sqrt{2}}\begin{bmatrix}1 \\  0\end{bmatrix} + \frac{1}{\sqrt{2}}\begin{bmatrix}0 \\  1\end{bmatrix})\\
                &= \frac{1}{\sqrt{2}}\ket{x} + \frac{1}{\sqrt{2}}\ket{y}
      \end{align*}

      So,

      \begin{align*}
        T_3 = \begin{bmatrix}
                &1 &0 &\\
                &\frac{1}{\sqrt{2}} &\frac{1}{\sqrt{2}} &
              \end{bmatrix}
      \end{align*}

      The transformation matrix $T_3$ can also be obtained by multiplying $T_1$ and $T_2$

      \begin{align*}
        T_2 \cdot T_1 &= \begin{bmatrix} &\frac{1}{\sqrt{2}} &\frac{1}{\sqrt{2}} &\\ &\frac{1+i}{2} &\frac{1-i}{2} & \end{bmatrix} \cdot \frac{1}{\sqrt{2}}\begin{bmatrix} &1 &-i &\\ &1 &i &\end{bmatrix} \\
        &= \begin{bmatrix}
              &\frac{1}{\sqrt{2}}\cdot\frac{1}{\sqrt{2}}+ \frac{1}{\sqrt{2}}\cdot\frac{1}{\sqrt{2}} & \frac{1}{\sqrt{2}}\cdot\frac{-i}{\sqrt{2}}+ \frac{1}{\sqrt{2}}\cdot\frac{i}{\sqrt{2}} &\\
              &\frac{1+i}{2}\cdot\frac{1}{\sqrt{2}}+ \frac{1-i}{2}\cdot\frac{1}{\sqrt{2}} & \frac{1+i}{2}\cdot\frac{-i}{\sqrt{2}}+ \frac{1-i}{2}\cdot\frac{i}{\sqrt{2}} &\\
           \end{bmatrix}\\
        &= \begin{bmatrix}
            & 1 &0 &\\
            &\frac{1}{\sqrt{2}} &\frac{1}{\sqrt{2}} &
           \end{bmatrix}\\
        &= T_3
      \end{align*}
    \end{answer}

  \end{enumerate}

  \item The probability that a photon in state $\ket{\psi}$ passes through an x-polarizer is the average value of a physical observable which could be called ``x-polarizedness”. Write down the operator, $P_x$, corresponding to this observable.

  \begin{answer}
    \begin{align*}
      P_x = \begin{bmatrix}
              & 1 &0 &\\
              &0  &0 &
            \end{bmatrix}
    \end{align*}
  \end{answer}

  \begin{enumerate}
    \item Show that it is Hermitian

    \begin{answer}
      For $P_x$ to be hermitian, it has to be equal to it's own conjugate transpose. Since it has no imaginary values, $P_x^* = P_x$. Since all off diagonal values are $0$, $P_x^T = P_x$. Therefore, $P_x = P_x^\dagger$
    \end{answer}

    \item Find its eigenvalues and eigenvectors.

    \begin{answer}
      Characteristic equation:

      \begin{align*}
        0 &= \abs{P_x -\lambda \mathbb{I}}\\
        0 &= \abs{\begin{bmatrix} &1 &0 &\\ &0 &0 & \end{bmatrix} - \begin{bmatrix} &\lambda &0 &\\ &0 &\lambda & \end{bmatrix}}\\
        0 &= \abs{\begin{bmatrix} &1 - \lambda &0 &\\ &0 &-\lambda & \end{bmatrix}}\\
        0 &= (1 - \lambda)\lambda\\
        \therefore \lambda_1 &= 0 \; \; \& \; \; \lambda_2 = 1
      \end{align*}

      So,

      \begin{align*}
        \phi_1 &= \begin{bmatrix}& 0 &\\ & 1 & \end{bmatrix}\\
        \phi_2 &= \begin{bmatrix}& 1 &\\ & 0 & \end{bmatrix}
      \end{align*}

    \end{answer}

    \item Write down its representation in the form

    \begin{align*}
      \lambda_1\ket{\phi_1}\bra{\phi_1} + \lambda_2\ket{\phi_2}\bra{\phi_2}
    \end{align*}

    \begin{answer}
      \begin{align*}
        \ket{x}\bra{x}
      \end{align*}
    \end{answer}

    \item Verify that the probability that a photon in state $\ket{\psi}$ passes through the x-polarizer is $\bra{\psi}P_x\ket{\psi}$

    \begin{answer}
      Let $\ket{psi}$ be an arbitrary photon

      \begin{align*}
        \ket{\psi} = \alpha\ket{x} + \beta\ket{y}
      \end{align*}

      The probability of this photon passing an x-polarized filter is

      \begin{align*}
          &\abs{\bra{x}\ket{\psi}}^2\\
        = &\abs{\bra{x}(\alpha\ket{x} + \beta\ket{y})}^2\\
        = &\abs{\bra{x}\alpha\ket{x} + \bra{x}\beta\ket{y}}^2\\
        = &\abs{\alpha\bra{x}\ket{x} + \beta\bra{x}\ket{y}}^2\\
        = &\abs{\alpha}^2
      \end{align*}

      Using the $P_x$ operator

      \begin{align*}
          &\bra{\psi}P_x\ket{\psi}\\
        = &(\bra{x}\alpha^* + \bra{y}\beta^*)\ket{x}\bra{x}(\alpha\ket{x} + \beta\ket{y})\\
        = &(\bra{x}\alpha^*\ket{x} + \bra{y}\beta^*\ket{x})(\bra{x}\alpha\ket{x} + \bra{x}\beta\ket{y})\\
        = &(\alpha^*\bra{x}\ket{x} + \beta^*\bra{y}\ket{x})(\alpha\bra{x}\ket{x} + \beta\bra{x}\ket{y})\\
        = &\alpha^*\alpha\\
        = &\abs{\alpha}^2
      \end{align*}
    \end{answer}
  \end{enumerate}

  \item A unitary operator $U$ is one satisfying

  \begin{align*}
    UU^\dagger = U^\dagger U = 1
  \end{align*}

  \begin{enumerate}
    \item Show that if $\bra{A}\ket{A} = 1$ and $\ket{B} = U \ket{A}$ then $\bra{B}\ket{B} = 1$

    \begin{answer}
      \begin{align*}
          &\bra{B}\ket{B}\\
        = &\bra{A}U^\dagger U\ket{A}\\
        = &\bra{A}1\ket{A}\\
        = &\bra{A}\ket{A}\\
        = &1
      \end{align*}
    \end{answer}

    \item Show that if $H$ is Hermitian, then $e^{iH}$ is unitary

    \begin{answer}
      \begin{align*}
          &(e^{iH})(e^{iH})^\dagger \\
        = &(e^{iH})(e^{-iH^\dagger}) \\
        = &e^{iH}e^{-iH} \\
        = &e^{iH-iH} \\
        = &e^{0} \\
        = &1
      \end{align*}
    \end{answer}

    \item Show that if $\ket{u_i}$ for $i=1 \dots N$ form a complete orthonormal set so that

    \begin{align*}
      \bra{u_i}\ket{u_j} = \delta_{ij}
    \end{align*}

    then the states $\ket{v_i} = U\ket{u_i}$ (for $i=1 \dots N$) also form an orthonormal set for unitary $U$.

    \begin{answer}
      \begin{align*}
          &\bra{v_i}\ket{v_j}\\
        = &\bra{u_i}U^\dagger U\ket{u_j}\\
        = &\bra{u_i}1\ket{u_j}\\
        = &\bra{u_i}\ket{u_j}\\
        = &\delta_{ij}
      \end{align*}
    \end{answer}
  \end{enumerate}

  \item Show that for a finite dimensional space, for any operators $X, Y, Z$

  \begin{enumerate}
    \item $\bra{B}X^\dagger\ket{A} = \bra{A}X\ket{B}^*$

    \begin{answer}

    \end{answer}

    \item $(X^\dagger)^\dagger = X$

    \begin{answer}
      let

      \begin{align*}
        X &= \begin{bmatrix}
                a_{11} & a_{12} & \dots  & a_{1j} &\\
                a_{21} & \ddots &        & \vdots &\\
                \vdots &        & \ddots & \vdots &\\
                a_{i1} & \dots  &  \dots & a_{ij} &
             \end{bmatrix}\\
        X^\dagger &= \begin{bmatrix}
                a_{11}^* & a_{21}^* & \dots  & a_{i1}^* &\\
                a_{12}^*  & \ddots  &        & \vdots &\\
                \vdots    &         & \ddots & \vdots &\\
                a_{1j}^*  & \dots  &  \dots  & a_{ij}^* &
            \end{bmatrix}\\
        (X^\dagger)^\dagger &= \begin{bmatrix}
                a_{11} & a_{12} & \dots  & a_{1j} &\\
                a_{21} & \ddots &        & \vdots &\\
                \vdots &        & \ddots & \vdots &\\
                a_{i1} & \dots  &  \dots & a_{ij} &
             \end{bmatrix}\\
      \end{align*}
    \end{answer}

    \item $(XY)^\dagger = Y^\dagger X^\dagger$

    \begin{answer}\label{daggerProof}
      Lets assume the existence of 2 wavefunctions $\bra{\psi_1}$ and $\ket{\psi_2}$

      \begin{align*}
          &\bra{\psi_1}(XY)^\dagger\ket{\psi_2}\\
        = &\bra{\psi_2}XY\ket{\psi_1}^*\\
        = &\sum_c\bra{\psi_2}X\ket{c}^*\bra{c}Y\ket{\psi_1}^*\\
        = &\sum_c\bra{c}X^\dagger\ket{\psi_2}\bra{\psi_1}Y^\dagger\ket{c}\\
        = &\sum_c\bra{\psi_1}Y^\dagger\ket{c}\bra{c}X^\dagger\ket{\psi_2}\\
        = &\bra{\psi_1}Y^\dagger X^\dagger\ket{\psi_2}\\
      \end{align*}
    \end{answer}

    \item $(XYZ)^\dagger = Z^\dagger Y^\dagger X^\dagger$

    \begin{answer}
      Assuming answer from part \ref{daggerProof}.Let,
      \begin{align*}
        A &= XY\\
        B &= Z\mathbb{I}\\
        \therefore (XYZ)^\dagger &= (AB)^\dagger\\
        &= B^\dagger A^\dagger\\
        &= (Z\mathbb{I})^\dagger (XY)^\dagger \\
        &= \mathbb{I}^\dagger Z^\dagger Y^\dagger X^\dagger\\
        &= Z^\dagger Y^\dagger X^\dagger\\
      \end{align*}
    \end{answer}

    \item $(\ket{B}\bra{A})^\dagger = \ket{A}\bra{B}$

    \begin{answer}
      \begin{align*}
        \ket{A}\bra{B} &=\begin{bmatrix}
                            a_1b_1 & a_1b_2 & \dots  & a_1b_i &\\
                            a_2b_1 & a_2b_2 & \dots  & a_2b_i &\\
                            \vdots & \vdots & \ddots & \vdots &\\
                            a_ib_1 & a_ib_2 &  \dots & a_ib_i &
                         \end{bmatrix}\\
         \ket{B}\bra{A} &=\begin{bmatrix}
                             b_1a_1 & b_1a_2 & \dots  & b_1a_i &\\
                             b_2a_1 & b_2a_2 & \dots  & b_2a_i &\\
                             \vdots & \vdots & \ddots & \vdots &\\
                             b_ia_1 & b_ia_2 &  \dots & b_ia_i &
                          \end{bmatrix}\\
          (\ket{B}\bra{A})^\dagger &=\begin{bmatrix}
                                      b_1a_1 & b_2a_1 & \dots  & b_ia_1 &\\
                                      b_1a_2 & b_2a_2 & \dots  & b_ia_2 &\\
                                      \vdots & \vdots & \ddots & \vdots &\\
                                      b_1a_i & b_2a_i &  \dots & b_ia_i &
                                   \end{bmatrix}\\
      \end{align*}
    \end{answer}

  \end{enumerate}
\end{enumerate}

\end{document}
